\documentclass[11pt,preprint, authoryear]{elsarticle}

\usepackage{lmodern}
%%%% My spacing
\usepackage{setspace}
\setstretch{1.2}
\DeclareMathSizes{12}{14}{10}{10}

% Wrap around which gives all figures included the [H] command, or places it "here". This can be tedious to code in Rmarkdown.
\usepackage{float}
\let\origfigure\figure
\let\endorigfigure\endfigure
\renewenvironment{figure}[1][2] {
    \expandafter\origfigure\expandafter[H]
} {
    \endorigfigure
}

\let\origtable\table
\let\endorigtable\endtable
\renewenvironment{table}[1][2] {
    \expandafter\origtable\expandafter[H]
} {
    \endorigtable
}


\usepackage{ifxetex,ifluatex}
\usepackage{fixltx2e} % provides \textsubscript
\ifnum 0\ifxetex 1\fi\ifluatex 1\fi=0 % if pdftex
  \usepackage[T1]{fontenc}
  \usepackage[utf8]{inputenc}
\else % if luatex or xelatex
  \ifxetex
    \usepackage{mathspec}
    \usepackage{xltxtra,xunicode}
  \else
    \usepackage{fontspec}
  \fi
  \defaultfontfeatures{Mapping=tex-text,Scale=MatchLowercase}
  \newcommand{\euro}{€}
\fi

\usepackage{amssymb, amsmath, amsthm, amsfonts}

\def\bibsection{\section*{References}} %%% Make "References" appear before bibliography


\usepackage[round]{natbib}

\usepackage{longtable}
\usepackage[margin=2.3cm,bottom=2cm,top=2.5cm, includefoot]{geometry}
\usepackage{fancyhdr}
\usepackage[bottom, hang, flushmargin]{footmisc}
\usepackage{graphicx}
\numberwithin{equation}{section}
\numberwithin{figure}{section}
\numberwithin{table}{section}
\setlength{\parindent}{0cm}
\setlength{\parskip}{1.3ex plus 0.5ex minus 0.3ex}
\usepackage{textcomp}
\renewcommand{\headrulewidth}{0.2pt}
\renewcommand{\footrulewidth}{0.3pt}

\usepackage{array}
\newcolumntype{x}[1]{>{\centering\arraybackslash\hspace{0pt}}p{#1}}

%%%%  Remove the "preprint submitted to" part. Don't worry about this either, it just looks better without it:
\journal{Journal of Finance}

 \def\tightlist{} % This allows for subbullets!

\usepackage{hyperref}
\hypersetup{breaklinks=true,
            bookmarks=true,
            colorlinks=true,
            citecolor=blue,
            urlcolor=blue,
            linkcolor=blue,
            pdfborder={0 0 0}}


% The following packages allow huxtable to work:
\usepackage{siunitx}
\usepackage{multirow}
\usepackage{hhline}
\usepackage{calc}
\usepackage{tabularx}
\usepackage{booktabs}
\usepackage{caption}


\newenvironment{columns}[1][]{}{}

\newenvironment{column}[1]{\begin{minipage}{#1}\ignorespaces}{%
\end{minipage}
\ifhmode\unskip\fi
\aftergroup\useignorespacesandallpars}

\def\useignorespacesandallpars#1\ignorespaces\fi{%
#1\fi\ignorespacesandallpars}

\makeatletter
\def\ignorespacesandallpars{%
  \@ifnextchar\par
    {\expandafter\ignorespacesandallpars\@gobble}%
    {}%
}
\makeatother

\newenvironment{CSLReferences}[2]{%
}

\urlstyle{same}  % don't use monospace font for urls
\setlength{\parindent}{0pt}
\setlength{\parskip}{6pt plus 2pt minus 1pt}
\setlength{\emergencystretch}{3em}  % prevent overfull lines
\setcounter{secnumdepth}{5}

%%% Use protect on footnotes to avoid problems with footnotes in titles
\let\rmarkdownfootnote\footnote%
\def\footnote{\protect\rmarkdownfootnote}
\IfFileExists{upquote.sty}{\usepackage{upquote}}{}

%%% Include extra packages specified by user

%%% Hard setting column skips for reports - this ensures greater consistency and control over the length settings in the document.
%% page layout
%% paragraphs
\setlength{\baselineskip}{12pt plus 0pt minus 0pt}
\setlength{\parskip}{12pt plus 0pt minus 0pt}
\setlength{\parindent}{0pt plus 0pt minus 0pt}
%% floats
\setlength{\floatsep}{12pt plus 0 pt minus 0pt}
\setlength{\textfloatsep}{20pt plus 0pt minus 0pt}
\setlength{\intextsep}{14pt plus 0pt minus 0pt}
\setlength{\dbltextfloatsep}{20pt plus 0pt minus 0pt}
\setlength{\dblfloatsep}{14pt plus 0pt minus 0pt}
%% maths
\setlength{\abovedisplayskip}{12pt plus 0pt minus 0pt}
\setlength{\belowdisplayskip}{12pt plus 0pt minus 0pt}
%% lists
\setlength{\topsep}{10pt plus 0pt minus 0pt}
\setlength{\partopsep}{3pt plus 0pt minus 0pt}
\setlength{\itemsep}{5pt plus 0pt minus 0pt}
\setlength{\labelsep}{8mm plus 0mm minus 0mm}
\setlength{\parsep}{\the\parskip}
\setlength{\listparindent}{\the\parindent}
%% verbatim
\setlength{\fboxsep}{5pt plus 0pt minus 0pt}



\begin{document}



\begin{frontmatter}  %

\title{Question 4}

% Set to FALSE if wanting to remove title (for submission)




\author[Add1]{Vincent Reinshagen\footnote{\textbf{Contributions:}
  \newline \emph{The authors would like to thank no institution for
  money donated to this project. Thank you sincerely.}}}
\ead{vreinshagen@outlook.de}





\address[Add1]{Stellenbosch University, Stellenbosch, South Africa}

\cortext[cor]{Corresponding author: Vincent Reinshagen\footnote{\textbf{Contributions:}
  \newline \emph{The authors would like to thank no institution for
  money donated to this project. Thank you sincerely.}}}

\begin{abstract}
\small{
This readme explores India's historical performance in the Summer
Olympics compared to similarly sized economies, and emerging markets,
alongside analyses of dominant countries across both Summer and Winter
Games. It also delves into nations that exceed expectations in medal
counts relative to their size and resources.
}
\end{abstract}

\vspace{1cm}





\vspace{0.5cm}

\end{frontmatter}

\setcounter{footnote}{0}



%________________________
% Header and Footers
%%%%%%%%%%%%%%%%%%%%%%%%%%%%%%%%%
\pagestyle{fancy}
\chead{}
\rhead{}
\lfoot{}
\rfoot{\footnotesize Page \thepage}
\lhead{}
%\rfoot{\footnotesize Page \thepage } % "e.g. Page 2"
\cfoot{}

%\setlength\headheight{30pt}
%%%%%%%%%%%%%%%%%%%%%%%%%%%%%%%%%
%________________________

\headsep 35pt % So that header does not go over title




\hypertarget{introduction}{%
\section{\texorpdfstring{Introduction
\label{Introduction}}{Introduction }}\label{introduction}}

This readme delves into India's performance in the Summer Olympics
relative to similarly sized economies, emerging markets, and select
South American nations. It also provides analyses of dominant countries
across both Summer and Winter Games, and explores which nations
consistently exceed expectations in medal counts relative to their size
and resources. Additionally, personal insights are offered on a favorite
Olympic event, with analyses of historical winners and participating
countries.

To start, I will plot India's medal count compared to other emerging
markets. Despite its large size, high absolute GDP, and substantial
population, India's prerequisites for achieving success in this context
should be favorable.

\begin{figure}[H]

{\centering \includegraphics{Question4_files/figure-latex/Figure 1-1} 

}

\caption{Medals won by India compared to other ermeging economies}\label{fig:Figure 1}
\end{figure}

This initial visualization reveals that India has won very few medals
compared to its emerging economies competitors.

Smaller economies may excel in the Olympics due to their higher
concentration of resources per capita. To investigate this, I will plot
medal counts relative to population size.

\begin{figure}[H]

{\centering \includegraphics{Question4_files/figure-latex/Figure 2-1} 

}

\caption{Models for countries with high population}\label{fig:Figure 2}
\end{figure}

I will now conduct a comparative analysis between the Winter and Summer
Olympics, examining overall performance across all countries and all
years.

Lets investigate the most successful countries in the Summer Olympics

\begin{figure}[H]

{\centering \includegraphics{Question4_files/figure-latex/Figure 3-1} 

}

\caption{Top Countries Summer Olympics}\label{fig:Figure 3}
\end{figure}

To determine if a country exceeds its expected performance in the
Olympics, two key factors come into play: financial investment in
training Olympic athletes, which is costly, and population size, as
larger populations generally enhance the likelihood of success.

\begin{figure}[H]

{\centering \includegraphics{Question4_files/figure-latex/Figure 4-1} 

}

\caption{Medals per Income per GDP}\label{fig:Figure 4}
\end{figure}

\begin{figure}[H]

{\centering \includegraphics{Question4_files/figure-latex/Figure 5-1} 

}

\caption{Medals per Population}\label{fig:Figure 5}
\end{figure}

\bibliography{Tex/ref}





\end{document}
