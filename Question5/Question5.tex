\documentclass[11pt,preprint, authoryear]{elsarticle}

\usepackage{lmodern}
%%%% My spacing
\usepackage{setspace}
\setstretch{1.2}
\DeclareMathSizes{12}{14}{10}{10}

% Wrap around which gives all figures included the [H] command, or places it "here". This can be tedious to code in Rmarkdown.
\usepackage{float}
\let\origfigure\figure
\let\endorigfigure\endfigure
\renewenvironment{figure}[1][2] {
    \expandafter\origfigure\expandafter[H]
} {
    \endorigfigure
}

\let\origtable\table
\let\endorigtable\endtable
\renewenvironment{table}[1][2] {
    \expandafter\origtable\expandafter[H]
} {
    \endorigtable
}


\usepackage{ifxetex,ifluatex}
\usepackage{fixltx2e} % provides \textsubscript
\ifnum 0\ifxetex 1\fi\ifluatex 1\fi=0 % if pdftex
  \usepackage[T1]{fontenc}
  \usepackage[utf8]{inputenc}
\else % if luatex or xelatex
  \ifxetex
    \usepackage{mathspec}
    \usepackage{xltxtra,xunicode}
  \else
    \usepackage{fontspec}
  \fi
  \defaultfontfeatures{Mapping=tex-text,Scale=MatchLowercase}
  \newcommand{\euro}{€}
\fi

\usepackage{amssymb, amsmath, amsthm, amsfonts}

\def\bibsection{\section*{References}} %%% Make "References" appear before bibliography


\usepackage[round]{natbib}

\usepackage{longtable}
\usepackage[margin=2.3cm,bottom=2cm,top=2.5cm, includefoot]{geometry}
\usepackage{fancyhdr}
\usepackage[bottom, hang, flushmargin]{footmisc}
\usepackage{graphicx}
\numberwithin{equation}{section}
\numberwithin{figure}{section}
\numberwithin{table}{section}
\setlength{\parindent}{0cm}
\setlength{\parskip}{1.3ex plus 0.5ex minus 0.3ex}
\usepackage{textcomp}
\renewcommand{\headrulewidth}{0.2pt}
\renewcommand{\footrulewidth}{0.3pt}

\usepackage{array}
\newcolumntype{x}[1]{>{\centering\arraybackslash\hspace{0pt}}p{#1}}

%%%%  Remove the "preprint submitted to" part. Don't worry about this either, it just looks better without it:
\journal{Journal of Finance}

 \def\tightlist{} % This allows for subbullets!

\usepackage{hyperref}
\hypersetup{breaklinks=true,
            bookmarks=true,
            colorlinks=true,
            citecolor=blue,
            urlcolor=blue,
            linkcolor=blue,
            pdfborder={0 0 0}}


% The following packages allow huxtable to work:
\usepackage{siunitx}
\usepackage{multirow}
\usepackage{hhline}
\usepackage{calc}
\usepackage{tabularx}
\usepackage{booktabs}
\usepackage{caption}


\newenvironment{columns}[1][]{}{}

\newenvironment{column}[1]{\begin{minipage}{#1}\ignorespaces}{%
\end{minipage}
\ifhmode\unskip\fi
\aftergroup\useignorespacesandallpars}

\def\useignorespacesandallpars#1\ignorespaces\fi{%
#1\fi\ignorespacesandallpars}

\makeatletter
\def\ignorespacesandallpars{%
  \@ifnextchar\par
    {\expandafter\ignorespacesandallpars\@gobble}%
    {}%
}
\makeatother

\newenvironment{CSLReferences}[2]{%
}

\urlstyle{same}  % don't use monospace font for urls
\setlength{\parindent}{0pt}
\setlength{\parskip}{6pt plus 2pt minus 1pt}
\setlength{\emergencystretch}{3em}  % prevent overfull lines
\setcounter{secnumdepth}{5}

%%% Use protect on footnotes to avoid problems with footnotes in titles
\let\rmarkdownfootnote\footnote%
\def\footnote{\protect\rmarkdownfootnote}
\IfFileExists{upquote.sty}{\usepackage{upquote}}{}

%%% Include extra packages specified by user

%%% Hard setting column skips for reports - this ensures greater consistency and control over the length settings in the document.
%% page layout
%% paragraphs
\setlength{\baselineskip}{12pt plus 0pt minus 0pt}
\setlength{\parskip}{12pt plus 0pt minus 0pt}
\setlength{\parindent}{0pt plus 0pt minus 0pt}
%% floats
\setlength{\floatsep}{12pt plus 0 pt minus 0pt}
\setlength{\textfloatsep}{20pt plus 0pt minus 0pt}
\setlength{\intextsep}{14pt plus 0pt minus 0pt}
\setlength{\dbltextfloatsep}{20pt plus 0pt minus 0pt}
\setlength{\dblfloatsep}{14pt plus 0pt minus 0pt}
%% maths
\setlength{\abovedisplayskip}{12pt plus 0pt minus 0pt}
\setlength{\belowdisplayskip}{12pt plus 0pt minus 0pt}
%% lists
\setlength{\topsep}{10pt plus 0pt minus 0pt}
\setlength{\partopsep}{3pt plus 0pt minus 0pt}
\setlength{\itemsep}{5pt plus 0pt minus 0pt}
\setlength{\labelsep}{8mm plus 0mm minus 0mm}
\setlength{\parsep}{\the\parskip}
\setlength{\listparindent}{\the\parindent}
%% verbatim
\setlength{\fboxsep}{5pt plus 0pt minus 0pt}



\begin{document}



\begin{frontmatter}  %

\title{Question 5}

% Set to FALSE if wanting to remove title (for submission)




\author[Add1]{Vincent Reinshagen\footnote{\textbf{Contributions:}
  \newline \emph{The authors would like to thank no institution for
  money donated to this project. Thank you sincerely.}}}
\ead{vreinshagen@outlook.de}





\address[Add1]{Stellenbosch University, Stellenbosch, South Africa}

\cortext[cor]{Corresponding author: Vincent Reinshagen\footnote{\textbf{Contributions:}
  \newline \emph{The authors would like to thank no institution for
  money donated to this project. Thank you sincerely.}}}

\begin{abstract}
\small{
In this project, I will analyze the inflationary impact on the
quintessential South African braaibroodjie over a three-year period,
using data from Stats SA and daily scraped prices from a major retailer.
By constructing a Braaibroodjie index and visualizing price trends of
its key ingredients, I will create clear and compelling visualizations
to effectively communicate the inflation trends to a non-technical
audience, while also providing a detailed technical comparison between
the Braaibroodjie index and the official SA Inflation index for
potential publication by Stats SA.
}
\end{abstract}

\vspace{1cm}





\vspace{0.5cm}

\end{frontmatter}

\setcounter{footnote}{0}



%________________________
% Header and Footers
%%%%%%%%%%%%%%%%%%%%%%%%%%%%%%%%%
\pagestyle{fancy}
\chead{}
\rhead{}
\lfoot{}
\rfoot{\footnotesize Page \thepage}
\lhead{}
%\rfoot{\footnotesize Page \thepage } % "e.g. Page 2"
\cfoot{}

%\setlength\headheight{30pt}
%%%%%%%%%%%%%%%%%%%%%%%%%%%%%%%%%
%________________________

\headsep 35pt % So that header does not go over title




\hypertarget{introduction}{%
\section{\texorpdfstring{Introduction
\label{Introduction}}{Introduction }}\label{introduction}}

In this project, I will analyze the inflationary impact on the
quintessential South African braaibroodjie over a three-year period.
Utilizing detailed data from Stats SA and daily scraped prices from a
major retailer, I will construct a Braaibroodjie index that tracks the
price changes of essential ingredients: white bread, cheddar, margarine,
tomatoes, onions, salt, and chutney. The goal is to create clear and
compelling visualizations that make inflation trends easily
understandable for a non-technical audience. Key graphs and tables will
illustrate time-series trends, individual ingredient indexes, and
rolling correlations.

Additionally, I will provide a comprehensive technical analysis
comparing the Braaibroodjie index with the official SA Inflation index.
This includes summarizing the data, highlighting discrepancies, and
demonstrating the reliability and relevance of the constructed index.
The findings will be presented concisely to ensure clarity and impact,
with the aim of convincing Stats SA to publish the Braaibroodjie index.
The visualizations will be meticulously crafted to be both informative
and engaging for the upcoming segment on eNCA, where I will share these
insights with viewers.

To begin, I will create a plot to visualize the price changes for each
braaibroodjie ingredient over the three-year period. This initial
visualization will highlight the individual trends and inflationary
impacts on white bread, cheddar, margarine, tomatoes, onions, salt, and
chutney, providing a clear overview of how each product's price has
evolved.

\begin{figure}[H]

{\centering \includegraphics{Question5_files/figure-latex/Figure 1-1} 

}

\caption{Price changes of braaibroodje ingredients over time \label{Figure1}}\label{fig:Figure 1}
\end{figure}

This initial visualization reveals that all these prices have increased
over time.

Next, I will analyze how the overall price of a complete braaibroodjie
has changed over time. This will involve aggregating the prices of all
the ingredients to understand the cumulative inflationary impact,
revealing that a braaibroodjie would cost 100 rand more today than it
did four years ago.

\begin{figure}[H]

{\centering \includegraphics{Question5_files/figure-latex/Figure 2-1} 

}

\caption{Cost of producing a Braaibroodje over time \label{Figure1}}\label{fig:Figure 2}
\end{figure}

The braaibroodje index in the context of general consumer price trends
\label{Meth}

I will then compare my Braaibroodjie index to the official CPI index.
This comparison will help contextualize the specific inflationary impact
on the braaibroodjie within the broader framework of general consumer
price trends in South Africa.

\begin{figure}[H]

{\centering \includegraphics{Question5_files/figure-latex/Figure 3-1} 

}

\caption{Accuracy of the braaibrodje index \label{Figure1}}\label{fig:Figure 3}
\end{figure}

The results show that while the changes in the CPI and the Braaibroodjie
index are similar, the price fluctuations are more extreme for the
Braaibroodjie. This indicates that the specific ingredients of the
braaibroodjie have experienced greater volatility compared to the
overall basket of goods measured by the CPI.

\bibliography{Tex/ref}





\end{document}
